\documentclass{tufte-handout}
\usepackage{amsmath}
\usepackage[utf8]{inputenc}
\usepackage{mathpazo}
\usepackage{booktabs}
\usepackage{microtype}

\usepackage{listings}
\usepackage{color}

\definecolor{dkgreen}{rgb}{0,0.6,0}
\definecolor{gray}{rgb}{0.5,0.5,0.5}
\definecolor{mauve}{rgb}{0.58,0,0.82}

\lstset{frame=tb,
  language=Java,
  aboveskip=3mm,
  belowskip=3mm,
  showstringspaces=false,
  columns=flexible,
  basicstyle={\small\ttfamily},
  numbers=none,
  numberstyle=\tiny\color{gray},
  keywordstyle=\color{blue},
  commentstyle=\color{dkgreen},
  stringstyle=\color{mauve},
  breaklines=true,
  breakatwhitespace=true,
  tabsize=3
}


\pagestyle{empty}


\title{Closest Pair Report}
\author{Kasper Hjort Berthelsen, Jesper Kato Jensen, Frederik Schou Madsen, Heidi Regina Schröder, Julie Schou Sørensen  \& Jon Voigt Tøttrup}

\begin{document}
  \maketitle
  

  \section{Results}

  Our implementation produces the expected results on all input--output file pairs. %except {\tt slotermeier-23-tsp.txt}, where our code reports distance 2.531 as the shortest distance.  This may be because of rounding errors, or differences between {\tt float} and {\tt double} on various machines.
%  \sidenote{%Complete the report by filling in your correct names,  filling in the parts marked $[\ldots]$,  and changing other parts wherever necessary.  For instance, if your implementation passes all tests, then write that.  Remove the sidenotes in your final hand-in.  }

The following table shows the closest pairs in the input files {\tt wc-instance-65534.txt}.
Here $n$ denotes the number of points in the input,
and $(u,v)$ denotes a closest pair of points at distance $\delta$.

  \bigskip\noindent
  \begin{tabular}{rrrr}\toprule
    $n$ 	& $u$ 			& $v$ 			& $\delta$ \\\midrule
	2 		& (0,5, -0,0)	& (-0,5, 0,0) 	& 1,0 \\
	6 		& (0,5, -0,0) 	& (-0,5, 0,0) 	& 1,0 \\
	14 		& (0,5, -0,0) 	& (-0,5, 0,0) 	& 1,0 \\
	30 		& (0,5, -0,0) 	& (-0,5, 0,0) 	& 1,0 \\
	62 		& (0,5, -0,0) 	& (-0,5, 0,0) 	& 1,0 \\
	126 	& (0,5, -0,0) 	& (-0,5, 0,0) 	& 1,0 \\
	254 	& (0,5, -0,0) 	& (-0,5, 0,0) 	& 1,0 \\
	1.022 	& (0,5, -0,0) 	& (-0,5, 0,0) 	& 1,0 \\
	4.094 	& (0,5, -0,0) 	& (-0,5, 0,0) 	& 1,0 \\
	16.382 	& (0,5, -0,0) 	& (-0,5, 0,0) 	& 1,0 \\
    65.534 	& (0.5, -0.0) 	& (-0.5, 0.0) 	& 1.0 \\
%    $[\ldots]$ 
\\\bottomrule
  \end{tabular}

  \section{Implementation details}
The implementation follows Section 5.4 of Kleinberg and Tardos, \emph{Algorithm Design}, Addison--Wesley 2006. It is a divide and conquer technique that divides the two-dimensional space and works by recurrence. 

We start by constructing the set of points $P=\{ p_1, ..., p_n\}$. Each point is implemented as a \verb#point#-class that has a name, a $x$-coordinate and a $y$-coordinate as well as a \verb#distance#-method. The total set of points is then implemented as an ArrayList of points.

After constructing $P$ we construct $P_x$ and $P_y$, that is, $P$ sorted by its $x$-coordinate and $y$-coordinate, respectively. This takes $n\log n$ time. We call the recurrence \verb#ClosestPairRec#, that divides the space in a left half, $Q$, and a right half, $R$. Again we construct sets that are sorted by $x$-coordinates and $y$-coordinates; $Q_x, Q_y, R_x$ and $R_y$. $Q_x$ and $R_x$ are found in linear time by looping though $P_x$, but $Q_y$ and $R_y$ are sorted again, which factors the fastest found running time by $n\log n$. 

For the comparison of the points in the middle of the two half, we construct the set of sorted points $S_y= \{s: x^*-\delta \ge s \ge x^*+\delta\}$, where $x^*$ is the right most point in $Q$. For each point $s$ we inspect the following 15 points in $S_y$, as explained (5.10) of Kleinberg and Tardos, \emph{Algorithm Design}, Addison--Wesley 2008. Below you find the corresponding part of our code. 
\begin{verbatim}
closestPair.p0 = P.get(0);
closestPair.p1 = P.get(1);
dmin = P.get(0).distance(P.get(1));

for(int i=0;i<P.size()-1;i++) {
    for(int j=1; j<Math.min(P.size()-i, 15); j++) {
        d = P.get(i).distance(P.get(i+j));
        if(d<dmin) {
            dmin = d;
            closestPair.p0 = P.get(i);
            closestPair.p1 = P.get(i+j);
        }
    }
}
return closestPair;
  \end{verbatim}
Because of the extra sorting in each recurrence step, our running time is $O(n\log^{2} n)$ for $n$ points.

\end{document}